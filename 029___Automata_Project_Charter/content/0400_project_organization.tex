\section{Project Organization}
\sffamily % Use sans-serif font for this section's text

This section outlines the key stakeholders, project structure, roles and responsibilities, and project management processes for the Automata Odoo ERP System Implementation project. Effective organization and clear responsibilities are crucial for project success.

\subsection{Automata Key Stakeholders}
The key stakeholders from Automata System Integration Solution (Automata) involved in this project are organized by department below:

\subsubsection{Administration }
\vspace{0.3em} % Add a little space before the table
% Removed Email Address column
\begin{tabularx}{\linewidth}{@{} >{\RaggedRight\arraybackslash}X >{\RaggedRight\arraybackslash}X >{\RaggedRight\arraybackslash}X @{}}
\toprule
\textbf{Full Name} & \textbf{Job Title} & \textbf{Role in Project} \\
\midrule
Eng. Waleed Mostafa & Chef Director & Project Sponsor  \\
\addlinespace
Hanaa Tharwat &  System Administrator & Automata Project Manager  \\
\bottomrule
\end{tabularx}
\vspace{1em} % Add some space after the list structure

\subsection{Scando Integrated Solutions Project Team}
The core project team from Scando Integrated Solutions includes:
\begin{itemize}
    \item \textbf{Sameh El Gohary:} Strategist
    \item \textbf{Ziad Almasry:} Project Manager
    \item \textbf{Noura Taha:} Product Owner
    % Additional Scando resources may be assigned as needed based on project phase requirements.
\end{itemize}

\subsection{Project Structure}
\subsubsection{Decision-Making Hierarchy}
The project will be governed by the following structure to ensure clear lines of authority and efficient decision-making:

\begin{itemize}
    \item \textbf{Project Sponsor (Automata):}
    \begin{itemize}
        \item Eng. Waleed Mostafa (IT Director, Automata)
    \end{itemize}
    \vspace{0.5em}
    \item \textbf{Steering Committee:} Responsible for strategic oversight, issue resolution, and ensuring project alignment with Automata's goals.
    \begin{itemize}
        \item \textit{Chair:} Eng. Waleed Mostafa (Automata)
        \item \textit{Automata Members:}
        \begin{itemize}
            \item Ahmed Thabet (Procurement and Contracts Director)
            \item Michel Yacoub (Director of Finance)
        \end{itemize}
        \item \textit{Scando Integrated Solutions Members:}
        \begin{itemize}
            \item Sameh El Gohary (Operational Leader)
            \item Ziad Almasry (Project Manager)
        \end{itemize}
        \item \textit{Odoo Representative (Advisory):}
        \begin{itemize}
            \item Mr. Fouad Shaalan  (Partnership Manager, Odoo Middle East DMCC -- as per original proposal)
        \end{itemize}
    \end{itemize}
    \vspace{0.5em}
    \item \textbf{Project Management Team (Joint):} Responsible for day-to-day project management, coordination, and execution.
    \begin{itemize}
        \item \textit{Automata Project Manager:} Dr. Hanaa Tharwat
        \item \textit{Scando Project Manager:} Ziad Almasry
    \end{itemize}
\end{itemize}

\subsection{Roles and Responsibilities}
\begin{itemize}
    \item \textbf{Project Sponsor (Eng. Waleed Mostafa, Automata):}
    \begin{itemize}
        \item Provides overall project authorization and champions the project within Automata.
        \item Secures necessary funding and resources for Automata.
        \item Makes final decisions on high-level project issues and strategic direction.
        \item Provides ultimate approval for project milestones and deliverables.
    \end{itemize}
    \vspace{0.5em}
    \item \textbf{Steering Committee:}
    \begin{itemize}
        \item Provides strategic guidance and oversight for the project.
        \item Monitors overall project progress against objectives, timeline, and budget.
        \item Resolves escalated issues and risks that cannot be resolved by the Project Management Team.
        \item Approves major changes to project scope, schedule, or budget.
        \item Ensures alignment of the project with Automata's overall business strategy.
    \end{itemize}
    \vspace{0.5em}
    \item \textbf{Automata Project Manager (Dr Hanaa Tharwat):}
    \begin{itemize}
        \item Acts as the primary point of contact for Automata regarding project matters.
        \item Manages and coordinates Automata's internal resources and activities.
        \item Facilitates communication between Automata stakeholders and the Scando project team.
        \item Ensures Automata's responsibilities and deliverables are met on time.
        \item Participates in project planning, monitoring, and status reporting.
        \item Helps in tracking Automata-side project tasks and documentation.
        \item Facilitates scheduling of meetings and workshops for the Automata team.

    \end{itemize}
    \vspace{0.5em}
    \item \textbf{Scando Project Manager (Ziad Almasry):}
    \begin{itemize}
        \item Acts as the primary point of contact for Scando Integrated Solutions.
        \item Develops and maintains the detailed project plan, schedule, and budget.
        \item Manages and coordinates the Scando project team and their deliverables.
        \item Monitors project progress, identifies risks and issues, and implements mitigation strategies.
        \item Ensures the quality and timeliness of Scando's deliverables.
        \item Leads project status meetings and provides regular reports to the Steering Committee.
    \end{itemize}
    \vspace{0.5em}
    \item \textbf{Scando Operational Leader (Sameh El Gohary):}
    \begin{itemize}
        \item Provides high-level operational oversight from Scando's perspective.
        \item Acts as an escalation point for critical project issues within Scando.
        \item Ensures alignment of the project with Scando's strategic objectives and resource allocation.
    \end{itemize}
    \vspace{0.5em}
    \item \textbf{Automata Process Owners ( D. Tamer Mostafa, Mohammed Diab):}
    \begin{itemize}
        \item Provide in-depth expertise in their respective business domains (Procurement, Finance \& Accounting).
        \item Represent their departments' needs and requirements.
        \item Make decisions regarding business process design and configuration within their areas.
        \item Drive user adoption and champion the new system within their departments.
        \item Participate in UAT and approve system functionalities for their respective areas.
    \end{itemize}
    \vspace{0.5em}
    \item \textbf{Scando Product Owner (Noura Taha):}
    \begin{itemize}
        \item Gathers and analyzes Automata's business requirements.
        \item Configures and customizes Odoo modules to meet agreed-upon requirements.
        \item Conducts workshops, demonstrations, and training sessions.
        \item Assists with data migration and system testing.
        \item Provides functional support during UAT and post-go-live.
    \end{itemize}
    \vspace{0.5em}
    \item \textbf{Automata Team Members (Ahmed Hussain, Mahmoud Azzam, etc.):}
    \begin{itemize}
        \item Provide subject matter expertise from their specific operational roles.
        \item Participate in testing activities as required.
        \item Support data validation and cleansing efforts.
        \item Adopt and utilize the new system in their daily work.
    \end{itemize}
\end{itemize}

\subsection{Issue Management}
Issue management is a process designed to address issues that may arise during the course of a project. Issues are always associated with some degree of risk to the project and therefore need to be assessed and resolved in a timely manner, either within or outside of the project boundaries. Issues need to be resolved in a consistent and disciplined manner to maintain the quality of the deliverables, as well as to control schedules and costs.

The Issue Management Process provides the mechanism to ensure that issues are properly identified and documented, escalated for management review, and resolved quickly and efficiently. It includes:
\begin{itemize}[label=\textbullet]
    \item Procedures for the identification, assignment and escalation of issues.
    \item The level of management that needs to be involved for escalation.
    \item Target timeline for issue resolution.
    \item The tracking of issues.
\end{itemize}
The process is designed to handle technical problems or issues as well as to address process, organisational and operational issues.

\subsubsection{Raising and Submitting an Issue}
\begin{itemize}
    \item Any project team member may raise project issues with a pre-designed Project Issue form.
    \item The originator must assign a tentative priority to the issue, together with a designated Issue Owner.
    \begin{itemize}[label=\textendash] % Using en-dash for sub-bullets for differentiation
        \item \textbf{Critical} -- presents an immediate and critical obstacle to project work and deadlines.
        \item \textbf{High} -- may impact critical deadlines or the quality of major deliverables.
        \item \textbf{Medium} -- may impact future, less critical deadlines or sub-components of a deliverable.
        \item \textbf{Low} -- has no direct impact on any deadlines or quality of deliverables.
    \end{itemize}
\end{itemize}

\subsubsection{Logging and Assigning an Issue}
\begin{itemize}
    \item The Steering Committee will review submitted issues and assign issues to the appropriate issue owner(s).
    \item Once assigned by the Steering Committee, the designated Project Manager (from Scando or Automata, as appropriate for the issue domain) will record the issues onto the Project Issue Log.
    \item The designated Project Manager responsible for the log will update the Project Issue Log with the appropriate status:
    \begin{itemize}[label=\textendash]
        \item \textbf{Received} -- any issue that has been submitted but not yet accepted as an Open issue.
        \item \textbf{Open} -- any issue that has been accepted as a valid issue and is still in progress.
        \item \textbf{In Progress} -- any issue that has had work started on either its resolution or analysis.
        \item \textbf{Deferred} -- any issue that has been deferred to be resolved at a later bring forward date for stated reasons or any issue that has a temporary solution with the proviso that the issue be brought forward at a later date.
        \item \textbf{Waiting Approval} -- any issue that has been resolved but is awaiting approval by the relevant Automata Process Owner or Project Manager.
        \item \textbf{Resolved} -- any issue that has been resolved to the project team satisfaction and approved.
    \end{itemize}
\end{itemize}

\subsubsection{Managing Issue Resolution}
\begin{itemize}
    \item The designated Project Manager will ensure that all stakeholders agree with the target resolution dates.
    \item The designated Project Manager will monitor the progress of outstanding issues within the following general guidelines:
    \begin{itemize}[label=\textendash]
        \item \textbf{High} -- within 3 working days
        \item \textbf{Medium} -- within 10 working days
        \item \textbf{Low} -- best effort basis
    \end{itemize}
    \item The issue priority may be changed for valid business, technical, logistical or timing reasons, with agreement from the Project Management Team.
    \item The designated Project Manager will assess the validity of any change request related to an issue in conjunction with all affected parties.
    \item The Automata Project Manager or Scando Project Manager (depending on issue ownership) will escalate high priority issues as they become overdue:
    \begin{itemize}[label=\textendash]
        \item 3 working days overdue -- Escalate to Issue Owner and other Project Manager.
        \item 5 working days overdue -- Escalate to Project Management Team (Joint) and relevant Process Owners.
        \item 10 working days overdue -- Escalate to Steering Committee.
    \end{itemize}
    \item Once the issue is resolved, the issue owner(s) will notify the designated Project Manager who will obtain agreement/approval from the individual who raised the issue or the relevant Automata stakeholder.
\end{itemize}

\subsubsection{Reporting Status}
\begin{itemize}
    \item The Scando Project Manager, in coordination with the Automata Project Manager, will track and update the progress status of all outstanding issues in the Project Issue Log.
    \item The Scando Project Manager will produce regular status reports (e.g., biweekly or as agreed) for open, overdue, and deferred issues for review by the Project Management Team and the Steering Committee.
    \item The most current list and status of open issues will be maintained in the Issues Management Log, accessible to relevant project stakeholders.
\end{itemize}

\subsection{Change Management}
The Change Management Process provides a mechanism to manage requests for changes to any project deliverables, including project scope and schedule. This process allows for change during the project’s life cycle but always puts it in the context of the latest project plan agreed upon between the project team and management and, in the case of contractors, as contractually agreed. The following change control procedures consist of a series of steps that allow change to be identified, evaluated, priced, and tracked through closure.
\begin{itemize}
    \item Change requests must be submitted, with a pre-designed change request form, to the Project Management Team (Joint: Automata and Scando Project Managers) for review and initial assessment.
    \item Once accepted as valid and assessed for impact, the Project Management Team must submit significant change requests (e.g., those impacting overall scope, budget, or timeline) to the Project Steering Committee for approval.
    \item The responsible Project Manager (typically Scando PM for overall plan) must refine the project plan to incorporate tasks and activities resulting from any approved changes.
    \item The designated Project Manager (typically Scando PM) must record all change requests and update the Change Request Log to reflect the status of each change request.
    \item The Project Management Team may approve minor changes (i.e., low impact on costs or time schedule not affecting key milestones) and will log and circulate them for information to the Steering Committee.
\end{itemize}

\subsection{Risk Management}
Risks are inherent in any project. A risk is defined as any factor that may potentially interfere with the successful completion of the project. The challenge is to manage risks with a process that is unique to the project and reflects its operational environment (i.e., resources, complexity, size, etc.). It is important to recognize that risks are not events that have occurred, but rather events that might occur that would adversely impact the project. Events that have occurred and are impacting the project are addressed in either the Issue Management Process and/or the Change Management Process.

Risk identification, risk action planning, and risk monitoring are key tools for successfully completing a project. Part of controlling a project during its execution life cycle is to have an established risk management process. The iterative risk management process commences as part of project planning and continues to evolve until the project closes out.

\subsubsection{Risk Identification}
Risk identification provides the project team the opportunity to alert management of potential risk factors before they become real threats to the project. Risks are listed, analyzed for probability of occurrence and potential impact on the project, and prioritized. Risk identification occurs at the beginning of the project and continues throughout the project’s life cycle.
\begin{itemize}
    \item The Project Management Team (Joint) will assemble an initial list of risk factors with impact analysis and priority assigned during the project planning phase. This will be maintained in a Risk Register.
    \item Any project team member may identify ‘new’ or additional risks at any time by reporting them to their respective Project Manager.
    \item In identifying project risks, the originator should provide information as follows (to be documented in the Risk Register):
    \begin{itemize}[label=\textendash]
        \item \textbf{Risk Identifier} -- the unique identifier for the risk statement.
        \item \textbf{Risk Sources} -- the focus area, risk factor category, and risk factors.
        \item \textbf{Risk Condition} -- the existing conditions that may negatively impact the project.
        \item \textbf{Risk Impact} -- the potential impact if the identified risk materializes (e.g., on scope, schedule, cost, quality).
        \item \textbf{Risk Probability} -- the likelihood that the risk will actually occur (e.g., Low, Medium, High).
        \item \textbf{Risk Exposure} -- the overall threat of the risk to the project (often calculated from Impact x Probability, used for ranking).
        \item \textbf{Risk Context} -- background information that serves to clarify the risk situation.
        \item \textbf{Related Risks} -- inter-dependent risks.
    \end{itemize}
    \item The originator must submit identified risks (e.g., via a Risk Management Control form or agreed communication channel) to the Project Management Team for review, logging, and action.
\end{itemize}

\subsubsection{Risk Action Planning}
Risk action planning produces plans for addressing each major risk item and co-ordinates individual risk plans with the overall project plan. Risk planning ensures that project schedules and/or cost estimates are adjusted to ensure that adequate time is appropriated to properly develop and execute risk mitigation measures when required.
\begin{itemize}
    \item The Project Management Team (Joint) will meet regularly (e.g., during project status meetings) to review existing and newly identified risks and to develop action plans to mitigate or respond to such risks (e.g., Avoid, Transfer, Mitigate, Accept).
    \item The Project Management Team will designate a risk owner (a project team member) responsible for managing the agreed-to risk response measures.
    \item The Scando Project Manager will adjust the master project plan, if necessary, to reflect time estimates and resources required for the execution of risk response measures.
\end{itemize}

\subsection{Communication Management}
Project Communication Management includes processes required to ensure timely and appropriate generation, collection, dissemination, storage, and ultimate disposition of project information. These communication processes provide essential links among people, ideas and information that are critical to the successful completion of the project. Each process may involve effort from one or more individuals or groups of individuals based on the needs of the project and each of which generally occurs at least once in the project life cycle. Although the following processes are presented as distinct and independent mechanisms, in practice they may overlap or interact in ways not detailed here.

\vspace{0.5em}
\begin{itemize}[leftmargin=*, itemsep=0.5em, labelwidth=!, labelindent=0pt]
    \item \textbf{Project Charter:}
    \begin{itemize}[label=\textbullet, leftmargin=*, itemsep=0.2em]
        \item \textit{Purpose:} Define the scope, objectives, project organization, and overall approach of the project.
        \item \textit{Description:} The Project Charter is the single point of reference on the project that should be read by all project team members, executives and anyone new to the project.
    \end{itemize}
    \item \textbf{Project Status Meeting:}
    \begin{itemize}[label=\textbullet, leftmargin=*, itemsep=0.2em]
        \item \textit{Purpose:} The purpose of the meeting is to track the progress of the project.
        \item \textit{Description:} The Project Steering Committee will meet monthly to review project status and to formulate direction and decisions when required. The agenda shall include (Progress, Issue, Risks, Changes).
    \end{itemize}
    \item \textbf{Issue Log / Documentation:}
    \begin{itemize}[label=\textbullet, leftmargin=*, itemsep=0.2em]
        \item \textit{Purpose:} Document issues that arise throughout the project which are considered significant and may impact the project or scope.
        \item \textit{Description:} The designated Project Manager will be the custodian of the Issue Log and will endeavor to maintain and update the status of each issue.
    \end{itemize}
    \item \textbf{Change Requests:}
    \begin{itemize}[label=\textbullet, leftmargin=*, itemsep=0.2em]
        \item \textit{Purpose:} Identify changes to project scope and submit requests to Project Steering Committee for review and approval.
        \item \textit{Description:} Track changes that affect scope, budget, resources, deliverables and timelines.
    \end{itemize}
    \item \textbf{User Group Orientation Sessions:}
    \begin{itemize}[label=\textbullet, leftmargin=*, itemsep=0.2em]
        \item \textit{Purpose:} The various ERP System user groups will need to be communicated with in a specific fashion as the implementation project progresses.
        \item \textit{Description:} As each Phase of the Implementation schedule begins, the affected users will be informed of the details, scope, process, and communication procedures for the Phase.
    \end{itemize}
    \item \textbf{Scando University / Automata Knowledge Base:}
    \begin{itemize}[label=\textbullet, leftmargin=*, itemsep=0.2em]
        \item \textit{Purpose:} As Automata users learn the ERP System software, a supportive learning environment will become conducive to a healthy implementation process.
        \item \textit{Description:} Learning the applications will be the most important objective, but users will also examine and potentially reengineer existing Automata business processes. An educational platform/knowledge base will be established for the purpose of sharing and learning the users the internal processes and system usability.
    \end{itemize}
\end{itemize}
\newpage